\documentclass{article}
\usepackage{mymacro}
\usepackage{longtable,booktabs}
\usepackage[a4paper, margin=.7in]{geometry}
\usepackage{blindtext}

\def\tbs{\textbackslash}

\begin{document}

\section{Symbols}

Mainly based on physics package

\begin{center}
\begin{table}[htp]
    \begin{tabular}{|p{3cm}||l|l|p{6cm}|}
    \toprule
    Macro                     & Usage                            & Effect                   & Comments \\
    \midrule \midrule
    \tbs quantity             & \tbs qty(\tbs frac\{b\}\{a\})    & $\qty(\frac{b}{a})$      & Automatic bracing: (), [], \{\}, $\vert\vert$
    \\ \hline
                              & \tbs pqty\{a\}                   & $\pqty{a}$               & \tbs pqty: (); \tbs bqty: []; \tbs vqty: $\abs{}$; \tbs Bqty: \{\}
    \\ \hline \hline
    \textbf{Bracing}          &                                  &                          & * for no resize, manual bracing: \tbs big, \tbs Big, \tbs bigg, \tbs Bigg
    \\ \hline
    \tbs absolutevalue        & \tbs abs\{a\}                    & $\abs{a}$                & Absoluevalue
    \\ \hline
    \tbs norm                 & \tbs norm\{a\}                   & $\norm{a}$               & Norm
    \\ \hline
    \tbs opnorm               & \tbs opnorm\{a\}                 & $\opnorm{a}$             & Operator norm
    \\ \hline
    \tbs evaluated            & \tbs eval\{a\}\_0\^{}1           & $\eval{a}_0^1$           & Evaluation, also \tbs eval(a$\vert$ \_0\^{}1; \tbs eval[a$\vert$ \_0\^{}1
    \\ \hline
    \tbs order                & \tbs order\{a\}                  & $\order{a}$              & Order
    \\ \hline
    \tbs commutator           & \tbs comm\{a\}\{b\}              & $\comm{a}{b}$            & Commutator
    \\ \hline
    \tbs anticommutator       & \tbs acomm\{a\}\{b\}             & $\acomm{a}{b}$           & Anti-commutator
    \\ \hline
    \tbs poissonbracket       & \tbs pb\{a\}\{b\}                & $\pb{a}{b}$              & Poison bracket
    \\ \hline \hline
    \end{tabular}
    \begin{tabular}{|p{3cm}||l|l|p{6cm}|}
    \toprule
    \textbf{Vector}           &                                  &                          &
    \\ \hline
    \tbs vectorbold           & \tbs vb\{a\}                     & $\vb{a}$                 & Vector as bold (no Greek), * for italic and Greek
    \\ \hline
    \tbs vectorarrow          & \tbs va\{a\}                     & $\va{a}$                 & Vector with arrow (no Greek), * for italic and Greek
    \\ \hline
    \tbs vectorunit           & \tbs vu\{a\}                     & $\vu{a}$                 & With hat (no Greek), * for italic and Greek
    \\ \hline
    \tbs dotproduct           & \tbs vdot                        & $\vdot$                  & Dot product (bold cdot)
    \\ \hline
    \tbs crossproduct         & \tbs cross                       & $\cross$                 & or \tbs cp
    \\ \hline
    \tbs gradient             & \tbs grad\{a\}                   & $\grad{a}$               & Also valid for (), []. with \textit{arrowdel} command, it's changed to vector mode
    \\ \hline
    \tbs divergence           & \tbs div\{a\}                    & $\div{a}$                & Also valid for (), [].
    \\ \hline
    \tbs laplacian            & \tbs laplacian\{a\}              & $\laplacian{a}$          & Also valid for (), [].
    \\ \hline \hline
    \end{tabular}
\end{table}
\end{center}
\begin{longtable}{l||l|l|p{6cm}}
\textbf{Derivatives}      &                                  &                          &
\\ \hline
\tbs differential         & \tbs dd\{a\}                     & $\dd{a}$                 & Differential symbol; also valid for ()
\\ \hline
                          & \tbs dd[3]\{a\}                  & $\dd[3]{a}$              & Power
\\ \hline
\tbs variation            & \tbs var[3]\{a\}                 & $\var[3]{a}$             & Variation of functional; works as \tbs dd.
\\ \hline
\tbs derivative           & \tbs dv[2]\{a\}                  & $\dv[2]{a}$              & Derivative, powers available with []
\\ \hline
                          & \tbs dv\{f\}\{a\}                & $\dv{f}{a}$              & Two arguments
\\ \hline
                          & \tbs dv\{a\}(f)                  & $\dv{a}(f)$              & Low form
\\ \hline
                          & \tbs dv*\{f\}\{a\}               & $\dv*{f}{a}$             & Inline form
\\ \hline
\tbs partialderivative    & \tbs pdv\{f\}\{a\}               & $\pdv{f}{a}$             & Partial derivative. same to \tbs dv.
\\ \hline
                          & \tbs pdv\{f\}\{x\}\{y\}          & $\pdv{f}{x}{y}$          & Can take two variables
\\ \hline
\tbs functionalderivative & \tbs fdv\{F\}\{g\}               & $\fdv{F}{g}$             & Functional derivative; works as \tbs dv
\\ \hline \hline
\textbf{Dirac notation}   &                                  &                          & * for no resize
\\ \hline
\tbs ket                  & \tbs ket\{a\}                    & $\ket{a}$                & Ket
\\ \hline
\tbs bra                  & \tbs bra\{a\}                    & $\bra{a}$                & Bra
\\ \hline
                          & \tbs bra\{a\}\tbs ket\{b\}       & $\bra{a}\ket{b}$         & Auto contraction
\\ \hline
\tbs innerproduct         & \tbs braket\{a\}\{b\}            & $\braket{a}{b}$          & Braket. Also \tbs ip
\\ \hline
                          & \tbs braket\{a\}                 & $\braket{a}$             & Norm
\\ \hline
\tbs outerproduct         & \tbs ketbra\{a\}\{b\}            & $\ketbra{a}{b}$          & Outer, also \tbs op or \tbs dyad
\\ \hline
\tbs expectationvalue     & \tbs expval\{A\}                 & $\expval{A}$             & Expectation value (implicit), also \tbs ev. (Resize doesn't include A, ** to include)
\\ \hline
                          & \tbs expval\{A\}\{n\}            & $\expval{A}{n}$          & Expectation value (explicit)
\\ \hline
\tbs matrixelement        & \tbs mel\{n\}\{A\}\{m\}          & $\mel{n}{A}{m}$          & Matrix element, also \tbs matrixel. (Resize doesn't include A, ** to include)
\\ \hline \hline
\textbf{Matrix}           &                                  &                          &
\\ \hline
\tbs matrixquantity       & \tbs mqty\{a\& b\tbs \tbs c\& d\}& $\begin{matrix}\mqty{a&b\\c&d}\end{matrix}$        & Matrix, can be grouped as elements in larger matrix. Also works with (), *(), [], $\norm{}$. \tbs pmqty: (); \tbs Pmqty:* (); \tbs bmqty: []; \tbs vmqty: $\norm{}$
\\ \hline
\tbs smallmatrixquantity  & \tbs smqty\{a\& b\tbs \tbs c\& d\}& $\begin{matrix}\smqty{a&b\\c&d}\end{matrix}$       & Small matrix, same as above
\\ \hline
\tbs matrixdeterminant    & \tbs mdet\{a\}                   & $\begin{matrix}\mdet{a}\end{matrix}$               & Determinant;
\\ \hline
                          & \tbs smdet\{a\}                  & $\begin{matrix}\smdet{a}\end{matrix}$              & Determinant, small version
\\ \hline
\tbs identitymatrix       & \tbs imat\{3\}                   & $\begin{matrix}\smqty(\imat{3})\end{matrix}$       & Identity Matrix
\\ \hline
\tbs xmatrix              & \tbs xmat\{x\}\{2\}\{3\}         & $\begin{matrix}\smqty{\xmat{x}{2}{3}}\end{matrix}$  & Matrix filled with $x$
\\ \hline
                          & \tbs xmat*\{a\}\{2\}\{3\}        & $\begin{matrix}\smqty{\xmat*{x}{2}{3}}\end{matrix}$ & * assign indices to elements
\\ \hline
\tbs zeromatrix           & \tbs zmat\{2\}\{3\}              & $\begin{matrix}\smqty{\zmat{2}{3}}\end{matrix}$    & Zero matrix
\\ \hline
\tbs paulimatrix          & \tbs pmat\{1\}                   & $\begin{matrix}\smqty{\pmat{1}}\end{matrix}$       & Pauli [0, 1, 2, 3] matrix
\\ \hline
\tbs diagonalmatrix       & \tbs dmat\{a, b\}                & $\begin{matrix}\smqty{\dmat{a,b}}\end{matrix}$     & Diagonal matrix, up to 8 elements, add [0] option to fill with 0. matrix can be inputted as entries as well.
\\ \hline
\tbs antidiagonalmatrix   & \tbs admat\{a,b\}                & $\begin{matrix}\smqty{\admat{a,b}}\end{matrix}$    & Anti-diagonal matrix, as above.
\\ \hline \hline
\textbf{Text in math mode}&                                  &                           & Insert text in math mode, including spacing. Special macros see table~\ref{tab:text}.
\\ \hline
\tbs qqtext               & 1 \tbs qq\{word\} 2              & $1\qq{word}2$             & with *, only include spacing at the end.
\end{longtable}

\begin{table}[hpb]
    \caption{Text in math mode}\label{tab:text}
    \centering
    \begin{tabular}{llll}
    \tbs qcc & \tbs qif & \tbs qthen & \tbs qotherwise \\
    \tbs qunless & \tbs qgiven & \tbs qusing & \tbs qassume \\
    \tbs qsince & \tbs qlet & \tbs qfor & \tbs qall \\
    \tbs qeven & \tbs qinteger & \tbs qand & \tbs qor \\
    \tbs qas & \tbs qin
    \end{tabular}
    \begin{tabular}{|llll}
    $1\qcc1$ & $1\qif1$ & $1\qthen1$ & $1\qotherwise1$ \\
    $1\qunless1$ & $1\qgiven1$ & $1\qusing1$ & $1\qassume1$ \\
    $1\qsince1$ & $1\qlet1$ & $1\qfor1$ & $1\qall1$ \\
    $1\qeven1$ & $1\qinteger1$ & $1\qand1$ & $1\qor1$ \\
    $1\qas1$ & $1\qin1$
    \end{tabular}
\end{table}

Other special functions:

The functions in Tab.~\ref{tab:func} can be used as \tbs sin[2](x):
$\sin[2](x)$, which handles the sizing and powers.
\begin{table}[hpb]
    \caption{Functions}\label{tab:func}
\centering
\begin{tabular}{llll}
\tbs sin & \tbs sinh & \tbs arcsin & \tbs asin \\
\tbs cos & \tbs cosh & \tbs arccos & \tbs acos \\
\tbs tan & \tbs tanh & \tbs arctan & \tbs atan \\
\tbs csc & \tbs csch & \tbs arccsc & \tbs acsc \\
\tbs sec & \tbs sech & \tbs arcsec & \tbs asec \\
\tbs cot & \tbs coth & \tbs arccot & \tbs acot \\
\tbs log & \tbs ln
\end{tabular}
\begin{tabular}{|llll}
$\sin$ & $\sinh$ & $\arcsin$ & $\asin$ \\
$\cos$ & $\cosh$ & $\arccos$ & $\acos$ \\
$\tan$ & $\tanh$ & $\arctan$ & $\atan$ \\
$\csc$ & $\csch$ & $\arccsc$ & $\acsc$ \\
$\sec$ & $\sech$ & $\arcsec$ & $\asec$ \\
$\cot$ & $\coth$ & $\arccot$ & $\acot$ \\
$\log$ &$ \ln$
\end{tabular}
\end{table}

The functions in Tab.~\ref{tab:limit} can be used as \tbs max[2]\{x\}:
$\max[2]{x}$, which handles the sizing and subscript (traditional typeset \tbs
max\_2 is still available).
In display mode, it is
$$\max[2]{x}$$
\begin{table}[htp]
    \caption{Limits}\label{tab:limit}
\centering
\begin{tabular}{lll}
\tbs max & \tbs min & \tbs lim \\
\tbs sup & \tbs inf & \tbs argmin \\
\tbs argmax
\end{tabular}
\begin{tabular}{|lll}
$\max$ & $\min$ & $\lim$ \\
$\sup$ & $\inf$ & $\argmin$ \\
$\argmax$
\end{tabular}
\end{table}

\pagebreak
The functions in Tab.~\ref{tab:ftext} can be used with automatic sizing of [],
(), and \{\}:
\begin{table}[htp]
    \caption{Function as text}\label{tab:ftext}
\centering
\begin{tabular}{lll}
\tbs exp    & \tbs det  & \tbs Pr \\
\tbs tr     & \tbs Tr   & \tbs Res\\
\end{tabular}
\begin{tabular}{|lll}
$\exp$      & $\det $   & $\Pr$     \\
$\tr$       & $\Tr$     & $\Res$    \\
\end{tabular}
\end{table}

Following functions can be used with conditions:
\begin{table}[htp]
    \caption{Conditional}\label{tab:cond}
\centering
\begin{tabular}{lll}
\tbs ave[a]\{b\}\{c\} & \tbs prob\{a\}\{b\} \\
\tbs entro\{a\}\{b\}  & \tbs KLdiv\{a\}\{b\}\\
\end{tabular}
\begin{tabular}{|lll}
$\displaystyle\ave[a]{b}{c}$  & $\prob{a}{b} $ \\
$\entro{a}{b} $               & $\KLdiv{a}{b} $\\
\end{tabular}
\end{table}

Following functions are provided as plain text:
\begin{table}[hpb]
\centering
\begin{tabular}{lll}
\tbs rank & \tbs erf & \tbs ker \\
\tbs deg  & \tbs gcd & \tbs hom
\end{tabular}
\begin{tabular}{|lll}
$\rank$ & $\erf$ & $\ker$ \\
$\deg$ & $\gcd$ & $\hom$
\end{tabular}
\end{table}

The following symbols are defined\linebreak[0]
\begin{table}[!hpb]
    \caption{Symbols}\label{tab:sym}
\centering
\begin{tabular}{llll}
\tbs ell & \tbs binary & \tbs complex & \tbs integer \\
\tbs real & \tbs natural & \tbs hilb  &
\end{tabular}
\begin{tabular}{|llll}
$\ell$ & $\binary$ & $\complex$ & $\integer$\\
$\real$ & $\natural$ & $\hilb$ &
\end{tabular}
\end{table}

Also, some special operators are provided:

\tbs principalvalue or \tbs pv\{f\}: $\pv{f}$, \tbs PV\{f\}: $\PV{f}$, \tbs Re:
$\Re{z}$, \tbs Im: $\Im{z}$.

\section{Marks and Colors}

Several shortcut for colors are provided with the \emph{xcolor} package:

\red{\tbs red}, \blue{\tbs blue}, \green{\tbs green}

Also the hyperrefs are colored:

\begin{enumerate}
    \item In document ref: \hyperref[dummylabel]{dummy target}\label{dummylabel}
    \item cite: \cite{dummy}
    \item url: \url{www.dummyurl}
\end{enumerate}

The marks are provided by the \emph{soul} package
\begin{enumerate}
    \item \tbs so: \so{space out}
    \item \tbs ul: \ul{underline} 
    \item \tbs st: \st{striking out}
    \item \tbs hl: \hl{highlight}
\end{enumerate}

\blindtext


\begin{thebibliography}{1}
    \bibitem{dummy} dummy citation
\end{thebibliography}

\end{document}
